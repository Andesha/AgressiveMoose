% Edit the title below to update the display in My Documents
\title{Modern OpenGL and Procedural Terrain Generation Exploration}

%
%%% Preamble
\documentclass[paper=a4, fontsize=11pt]{scrartcl}
\usepackage[T1]{fontenc}
\usepackage{fourier}

\usepackage[english]{babel}															% English language/hyphenation
\usepackage[protrusion=true,expansion=true]{microtype}	
\usepackage{amsmath,amsfonts,amsthm} % Math packages
\usepackage[pdftex]{graphicx}	
\usepackage{url}


%%% Custom sectioning
\usepackage{sectsty}
\allsectionsfont{\centering \normalfont\scshape}


%%% Custom headers/footers (fancyhdr package)
\usepackage{fancyhdr}
\pagestyle{fancyplain}
\fancyhead{}											% No page header
\fancyfoot[L]{}											% Empty 
\fancyfoot[C]{}											% Empty
\fancyfoot[R]{\thepage}									% Pagenumbering
\renewcommand{\headrulewidth}{0pt}			% Remove header underlines
\renewcommand{\footrulewidth}{0pt}				% Remove footer underlines
\setlength{\headheight}{13.6pt}


%%% Equation and float numbering
\numberwithin{equation}{section}		% Equationnumbering: section.eq#
\numberwithin{figure}{section}			% Figurenumbering: section.fig#
\numberwithin{table}{section}				% Tablenumbering: section.tab#


%%% Maketitle metadata
\newcommand{\horrule}[1]{\rule{\linewidth}{#1}} 	% Horizontal rule

\title{
		%\vspace{-1in} 	
		\usefont{OT1}{bch}{b}{n}
		\normalfont \normalsize \textsc{Brock University Computer Science 3P98} \\ [25pt]
		\horrule{0.5pt} \\[0.4cm]
		\huge Modern OpenGL and Procedural Terrain Generation Exploration \\
		\horrule{2pt} \\[0.5cm]
}
\author{
		\normalfont 								\normalsize
        Tyler Kennedy Collins\\[-3pt]				\normalsize
        tk11br@brocku.ca\\[-3pt]					\normalsize
        4956637\\									\normalsize
        \and
        Preston Engstrom\\[-3pt]					\normalsize
        pe12nh@brocku.ca\\[-3pt]					\normalsize
        5228549\\									\normalsize
}
\date{}


%%% Begin document
\begin{document}
\maketitle
\section{Introduction}
The main purpose of this project was to explore not only the use of modern OpenGL, (version 3.3) but to experiment with the process of developing a system which could procedurally generate terrain. An additional motivation was the ability to obtain experience working with third party libraries. Some examples from among these were glm, SDL, and SOIL. Included with our project is a small C\# based application which we used to explore the boundaries of Perlin Noise in relation to terrain generation. An additional two avenues that we explored were model loading, and engine building. Luckily for us, our implementation of a physics engine was simply just collision with a surface. On the other hand, our model loading system allows for us to import complex meshed objects with different materials.

%%% End document
\end{document}